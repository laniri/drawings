% Performance Analysis Template for Algorithm Documentation
% Includes complexity analysis, benchmarking, and validation methodologies

\documentclass{article}
\usepackage{amsmath}
\usepackage{amsfonts}
\usepackage{amssymb}
\usepackage{amsthm}
\usepackage{mathtools}
\usepackage{algorithm}
\usepackage{algorithmic}
\usepackage{booktabs}
\usepackage{array}

% Custom commands for performance analysis
\newcommand{\bigO}[1]{\mathcal{O}\left(#1\right)}
\newcommand{\bigTheta}[1]{\Theta\left(#1\right)}
\newcommand{\bigOmega}[1]{\Omega\left(#1\right)}
\newcommand{\littleo}[1]{o\left(#1\right)}
\newcommand{\littleomega}[1]{\omega\left(#1\right)}

\begin{document}

% 1. COMPUTATIONAL COMPLEXITY ANALYSIS
\section{Computational Complexity Analysis}

\subsection{Time Complexity}

\subsubsection{Best Case Analysis}
The best-case time complexity occurs when {{BEST_CASE_CONDITION}}:
\begin{equation}
T_{\text{best}}(n) = \bigO{{{BEST_CASE_COMPLEXITY}}}
\end{equation}

\textbf{Analysis:} {{BEST_CASE_ANALYSIS}}

\subsubsection{Average Case Analysis}
Under the assumption of {{AVERAGE_CASE_ASSUMPTION}}, the expected time complexity is:
\begin{equation}
T_{\text{avg}}(n) = \mathbb{E}[T(n)] = \bigTheta{{{AVERAGE_CASE_COMPLEXITY}}}
\end{equation}

\textbf{Derivation:}
\begin{align}
\mathbb{E}[T(n)] &= \sum_{i} P(i) \cdot T_i(n) \\
&= {{AVERAGE_CASE_DERIVATION}} \\
&= \bigTheta{{{AVERAGE_CASE_COMPLEXITY}}}
\end{align}

\subsubsection{Worst Case Analysis}
The worst-case scenario occurs when {{WORST_CASE_CONDITION}}:
\begin{equation}
T_{\text{worst}}(n) = \bigO{{{WORST_CASE_COMPLEXITY}}}
\end{equation}

\textbf{Proof:} {{WORST_CASE_PROOF}}

\subsection{Space Complexity}

\subsubsection{Auxiliary Space}
The algorithm requires auxiliary space for:
\begin{align}
S_{\text{variables}} &= \bigO{{{VARIABLE_SPACE}}} \\
S_{\text{data structures}} &= \bigO{{{DATA_STRUCTURE_SPACE}}} \\
S_{\text{recursion}} &= \bigO{{{RECURSION_SPACE}}}
\end{align}

Total auxiliary space: $S_{\text{aux}}(n) = \bigO{{{TOTAL_AUXILIARY_SPACE}}}$

\subsubsection{Total Space}
Including input space, the total space complexity is:
\begin{equation}
S_{\text{total}}(n) = \bigO{{{TOTAL_SPACE_COMPLEXITY}}}
\end{equation}

% 2. SCALABILITY ANALYSIS
\section{Scalability Analysis}

\subsection{Input Size Scaling}

The algorithm's performance scales with input size $n$ as follows:

\begin{table}[h]
\centering
\begin{tabular}{@{}lcc@{}}
\toprule
Input Size & Time Complexity & Space Complexity \\
\midrule
$n \leq 10^3$ & $\bigO{{{SMALL_INPUT_TIME}}}$ & $\bigO{{{SMALL_INPUT_SPACE}}}$ \\
$10^3 < n \leq 10^6$ & $\bigO{{{MEDIUM_INPUT_TIME}}}$ & $\bigO{{{MEDIUM_INPUT_SPACE}}}$ \\
$n > 10^6$ & $\bigO{{{LARGE_INPUT_TIME}}}$ & $\bigO{{{LARGE_INPUT_SPACE}}}$ \\
\bottomrule
\end{tabular}
\caption{Scalability characteristics by input size}
\end{table}

\subsection{Parallel Scalability}

For $p$ parallel processors, the parallel time complexity is:
\begin{equation}
T_p(n) = \frac{T(n)}{p} + T_{\text{communication}}(n, p)
\end{equation}

where $T_{\text{communication}}(n, p) = \bigO{{{COMMUNICATION_COMPLEXITY}}}$.

The parallel efficiency is:
\begin{equation}
E(n, p) = \frac{T(n)}{p \cdot T_p(n)} = \frac{{{SEQUENTIAL_TIME}}}{{{PARALLEL_TIME}}}
\end{equation}

% 3. PERFORMANCE BENCHMARKS
\section{Performance Benchmarks}

\subsection{Benchmark Methodology}

\subsubsection{Test Environment}
\begin{itemize}
\item \textbf{Hardware:} {{HARDWARE_SPECIFICATIONS}}
\item \textbf{Software:} {{SOFTWARE_SPECIFICATIONS}}
\item \textbf{Compiler:} {{COMPILER_SPECIFICATIONS}}
\item \textbf{Optimization:} {{OPTIMIZATION_FLAGS}}
\end{itemize}

\subsubsection{Dataset Characteristics}
\begin{align}
\text{Dataset size:} &\quad n = {{DATASET_SIZE}} \\
\text{Feature dimension:} &\quad d = {{FEATURE_DIMENSION}} \\
\text{Data type:} &\quad {{DATA_TYPE}} \\
\text{Distribution:} &\quad {{DATA_DISTRIBUTION}}
\end{align}

\subsection{Execution Time Benchmarks}

\begin{table}[h]
\centering
\begin{tabular}{@{}lcccc@{}}
\toprule
Dataset Size & Mean Time (ms) & Std Dev (ms) & Min Time (ms) & Max Time (ms) \\
\midrule
{{#BENCHMARK_RESULTS}}
{{DATASET_SIZE}} & {{MEAN_TIME}} & {{STD_DEV}} & {{MIN_TIME}} & {{MAX_TIME}} \\
{{/BENCHMARK_RESULTS}}
\bottomrule
\end{tabular}
\caption{Execution time benchmarks across different dataset sizes}
\end{table}

\subsection{Memory Usage Benchmarks}

Peak memory usage follows the pattern:
\begin{equation}
M_{\text{peak}}(n) = {{MEMORY_FORMULA}}
\end{equation}

\begin{table}[h]
\centering
\begin{tabular}{@{}lccc@{}}
\toprule
Dataset Size & Peak Memory (MB) & Average Memory (MB) & Memory Efficiency \\
\midrule
{{#MEMORY_RESULTS}}
{{DATASET_SIZE}} & {{PEAK_MEMORY}} & {{AVERAGE_MEMORY}} & {{MEMORY_EFFICIENCY}} \\
{{/MEMORY_RESULTS}}
\bottomrule
\end{tabular}
\caption{Memory usage benchmarks}
\end{table}

% 4. ACCURACY AND QUALITY METRICS
\section{Accuracy and Quality Metrics}

\subsection{Accuracy Analysis}

The algorithm achieves the following accuracy metrics:

\begin{align}
\text{Precision} &= \frac{TP}{TP + FP} = {{PRECISION_VALUE}} \\
\text{Recall} &= \frac{TP}{TP + FN} = {{RECALL_VALUE}} \\
\text{F1-Score} &= \frac{2 \cdot \text{Precision} \cdot \text{Recall}}{\text{Precision} + \text{Recall}} = {{F1_SCORE}}
\end{align}

\subsection{Error Analysis}

\subsubsection{Mean Squared Error}
\begin{equation}
\text{MSE} = \frac{1}{n} \sum_{i=1}^{n} (y_i - \hat{y}_i)^2 = {{MSE_VALUE}}
\end{equation}

\subsubsection{Mean Absolute Error}
\begin{equation}
\text{MAE} = \frac{1}{n} \sum_{i=1}^{n} |y_i - \hat{y}_i| = {{MAE_VALUE}}
\end{equation}

\subsubsection{Root Mean Squared Error}
\begin{equation}
\text{RMSE} = \sqrt{\text{MSE}} = {{RMSE_VALUE}}
\end{equation}

% 5. COMPARATIVE ANALYSIS
\section{Comparative Analysis}

\subsection{Algorithm Comparison}

\begin{table}[h]
\centering
\begin{tabular}{@{}lcccc@{}}
\toprule
Algorithm & Time Complexity & Space Complexity & Accuracy & Implementation \\
\midrule
{{ALGORITHM_NAME}} & $\bigO{{{OUR_TIME}}}$ & $\bigO{{{OUR_SPACE}}}$ & {{OUR_ACCURACY}} & {{OUR_COMPLEXITY}} \\
{{#COMPARISON_ALGORITHMS}}
{{ALG_NAME}} & $\bigO{{{ALG_TIME}}}$ & $\bigO{{{ALG_SPACE}}}$ & {{ALG_ACCURACY}} & {{ALG_COMPLEXITY}} \\
{{/COMPARISON_ALGORITHMS}}
\bottomrule
\end{tabular}
\caption{Comparative analysis with existing algorithms}
\end{table}

\subsection{Performance Trade-offs}

The algorithm exhibits the following trade-offs:

\begin{itemize}
\item \textbf{Time vs. Accuracy:} {{TIME_ACCURACY_TRADEOFF}}
\item \textbf{Space vs. Time:} {{SPACE_TIME_TRADEOFF}}
\item \textbf{Scalability vs. Precision:} {{SCALABILITY_PRECISION_TRADEOFF}}
\end{itemize}

% 6. OPTIMIZATION ANALYSIS
\section{Optimization Analysis}

\subsection{Optimization Techniques Applied}

\subsubsection{{{OPTIMIZATION_1_NAME}}}
\textbf{Description:} {{OPTIMIZATION_1_DESCRIPTION}}

\textbf{Performance Impact:}
\begin{align}
\text{Time improvement:} &\quad {{TIME_IMPROVEMENT_1}} \\
\text{Space improvement:} &\quad {{SPACE_IMPROVEMENT_1}}
\end{align}

\subsubsection{{{OPTIMIZATION_2_NAME}}}
\textbf{Description:} {{OPTIMIZATION_2_DESCRIPTION}}

\textbf{Performance Impact:}
\begin{align}
\text{Time improvement:} &\quad {{TIME_IMPROVEMENT_2}} \\
\text{Space improvement:} &\quad {{SPACE_IMPROVEMENT_2}}
\end{align}

\subsection{Optimization Potential}

Theoretical lower bounds suggest:
\begin{align}
T_{\text{optimal}}(n) &\geq \bigOmega{{{THEORETICAL_LOWER_BOUND}}} \\
S_{\text{optimal}}(n) &\geq \bigOmega{{{SPACE_LOWER_BOUND}}}
\end{align}

Current implementation achieves {{OPTIMALITY_PERCENTAGE}}\% of theoretical optimum.

% 7. VALIDATION METHODOLOGY
\section{Validation Methodology}

\subsection{Cross-Validation}

\subsubsection{K-Fold Cross-Validation}
Using $k = {{K_VALUE}}$ folds, the cross-validation results are:

\begin{align}
\text{Mean Accuracy} &= \frac{1}{k} \sum_{i=1}^{k} A_i = {{CV_MEAN_ACCURACY}} \\
\text{Standard Deviation} &= \sqrt{\frac{1}{k-1} \sum_{i=1}^{k} (A_i - \bar{A})^2} = {{CV_STD_ACCURACY}}
\end{align}

\subsubsection{Leave-One-Out Cross-Validation}
LOOCV results: Mean accuracy = {{LOOCV_ACCURACY}}, Variance = {{LOOCV_VARIANCE}}

\subsection{Statistical Significance Testing}

\subsubsection{Paired t-test}
Comparing with baseline algorithm:
\begin{align}
H_0: &\quad \mu_{\text{diff}} = 0 \\
H_1: &\quad \mu_{\text{diff}} \neq 0
\end{align}

Test statistic: $t = \frac{\bar{d}}{s_d/\sqrt{n}} = {{T_STATISTIC}}$

p-value: $p = {{P_VALUE}}$

\subsubsection{Wilcoxon Signed-Rank Test}
Non-parametric test results:
\begin{itemize}
\item Test statistic: $W = {{WILCOXON_STATISTIC}}$
\item p-value: $p = {{WILCOXON_P_VALUE}}$
\item Effect size: $r = {{EFFECT_SIZE}}$
\end{itemize}

% 8. ROBUSTNESS ANALYSIS
\section{Robustness Analysis}

\subsection{Noise Sensitivity}

Performance under different noise levels:

\begin{table}[h]
\centering
\begin{tabular}{@{}lccc@{}}
\toprule
Noise Level & Accuracy Drop & Time Increase & Space Increase \\
\midrule
{{#NOISE_LEVELS}}
{{NOISE_LEVEL}} & {{ACCURACY_DROP}} & {{TIME_INCREASE}} & {{SPACE_INCREASE}} \\
{{/NOISE_LEVELS}}
\bottomrule
\end{tabular}
\caption{Performance degradation under noise}
\end{table}

\subsection{Parameter Sensitivity}

Sensitivity analysis for key parameters:

\begin{align}
\frac{\partial \text{Accuracy}}{\partial \theta_1} &= {{SENSITIVITY_PARAM_1}} \\
\frac{\partial \text{Accuracy}}{\partial \theta_2} &= {{SENSITIVITY_PARAM_2}}
\end{align}

% 9. CONFIDENCE INTERVALS
\section{Confidence Intervals}

\subsection{Performance Confidence Intervals}

95\% confidence intervals for key metrics:

\begin{align}
\text{Accuracy:} &\quad [{{ACCURACY_CI_LOWER}}, {{ACCURACY_CI_UPPER}}] \\
\text{Execution Time:} &\quad [{{TIME_CI_LOWER}}, {{TIME_CI_UPPER}}] \text{ ms} \\
\text{Memory Usage:} &\quad [{{MEMORY_CI_LOWER}}, {{MEMORY_CI_UPPER}}] \text{ MB}
\end{align}

\subsection{Bootstrap Confidence Intervals}

Using {{BOOTSTRAP_SAMPLES}} bootstrap samples:

\begin{align}
\text{Bootstrap Mean:} &\quad {{BOOTSTRAP_MEAN}} \\
\text{Bootstrap Std:} &\quad {{BOOTSTRAP_STD}} \\
\text{95\% CI:} &\quad [{{BOOTSTRAP_CI_LOWER}}, {{BOOTSTRAP_CI_UPPER}}]
\end{align}

\end{document}